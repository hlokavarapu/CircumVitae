%        10        20        30        40        50        60       70        80        90
%23456789012345678901234567890123456789012345678901234567890123456789012345678901234567890123456

% \documentclass[11pt]{article} %\documentclass[final,10pt,reqno, oneside,dvipsnames]{article}

% WARNING: Use of 'twoside' will cause \marginpar to flip from right to left margin, depending
% on the page.

\documentclass[11pt]{ltxdoc}

%%%%%%%%%%%%%%%%%%%%%%%%%%%%%%%%%%%%%%%%%%%%%%%%%%%%%%%%%%%%%%%%%%%%%%%%%%%%%%%%%%%%%%%%%%%%%%%% Template provided by PROFESSOR ELBRIDGE GERRY PUCKETT
%%%%%%%%%%%%%%%%%%% Harsha Lokavarapu CURRICULUM VITAE %%%%%%%%%%%%%%%%%%%%%%%%
%%%%%%%%%%%%%%%%%%%%%%%%%%%%%%%%%%%%%%%%%%%%%%%%%%%%%%%%%%%%%%%%%%%%%%%%%%%%%%%%%%%%%%%%%%%%%%%%


% PROFESSOR ELBRIDGE GERRY PUCKETT
% DEPARTMENT OF MATHEMATICS
% UNIVERSITY OF CALIFORNIA, DAVIS
%
% Original LaTeX file created by
%
%   Wednesday, July 04, 2018 08:49:51 PDT
%
%
% REVISION HISTORY:
%
%   Revision 1.00
%
%     Revision 1.00: Wednesday, July   04, 2018 08:49:51 PDT

%
%
%%%%%%%%%%%%%%%%%%%%%%%%%%%%%%%%%% TURN OFF BibTeX? %%%%%%%%%%%%%%%%%%%%%%%%%%%%%%%%%%%%%%%%%%%%

% If there is no need for BibTeX in this document turn it off, since it creates unwanted 
% files.

\let\nobibtex = t

%%%%%%%%%%%%%%%%%%%%%%%%%%%%%%%%% SET UP PAGE PARAMETERS %%%%%%%%%%%%%%%%%%%%%%%%%%%%%%%%%%%%%%%

% EGP's preferred page style for notes, homework assignments, exams, etc.

% Margins, paragraph indents, space between paragraphs if any, etc. Good references 
% include page 85 of "The LaTeX Companion" by Frank Mittelbach and Michel Goossens and 
% page 260 of "Math Into LaTeX" by George Gratzer.

% Enlarge the width and height of the printed page

\setlength{\textwidth }{7.50 in}
\setlength{\textheight}{9.50 in} % \textheight     = 9.25 in

% Space between the end of the odd and even side margins and the beginning of the text.

\setlength{\oddsidemargin }{0.0 in}
\setlength{\evensidemargin}{0.0 in}

% The side margins are 1.0 inch plus \hoffset and the top margins is 1.0 inch plus 
% \voffset. According to "The LaTeX Companion" the default values are \hoffset = 00 pt 
% and \voffset = 00 pt.

\setlength{\hoffset}{-0.50 in}
\setlength{\voffset}{-0.50 in}

% If there is no header in this document, set each of the following values to zero.

% This is the amount of space between the BOTTOM of the HEAD and the TOP of the BODY. 
% According to "Math Into LaTeX"  by George Gratzer the default values for LaTeX's 
% article document is \headsep = 25 pt.

\setlength{\headsep}{12 pt}

% Package Fancyhdr Warning: \headheight = 12 pt is too small. Make it at least 14.0pt. 
% According to "Math Into LaTeX"  by George Gratzer the default value for LaTeX's article 
% document is \headheight = 12 pt.

\setlength{\headheight}{14 pt}

% This is the amount of space between the top of the page and the TOP of the header. According to "Math
% Into LaTeX"  by George Gratzer the default value for LaTeX's article document is \topmargin = 16 pt.
% \headheight = 12pt, and .

\setlength{\topmargin }{00 pt}

% This is the amount of space between the TOP OF THE BODY and THE BOTTOM OF THE FIRST 
% line of text on the page. According to "Math Into LaTeX" by George Gratzer the default
% value for LaTeX's article documents is \topskip = ?? pt.

\setlength{\topskip}{12 pt}

% This is the amount of space between the last line on the page and the footer. According 
% to "The LaTeX Companion" and "Math Into LaTeX" the default value is \footskip = 30 pt.

\setlength{\footskip}{18 pt}

% I like space between paragraphs, since it makes the document more readable. However, 
% this does not seem to change the spacing between paragraphs contained in an item of a 
% list.

\setlength{\parskip}{12 pt} % default: \parskip        = 12  pt

% Comment out the following line to use the default amount to indent the first line of 
% each paragraph.

\setlength{\parindent}{00 pt}

% Add an underline of text that breaks the line properly with \uline{text} ...

\usepackage[normalem]{ulem}

%%%%%%%%%%%%%%%%%%%%%%%%%%%%%%% LOAD THE "enumitem" PACKAGE %%%%%%%%%%%%%%%%%%%%%%%%%%%%%%

\usepackage{enumitem}

% \setenumerate[1]{labelindent=0pt,itemindent=12pt}

\setlength{\labelwidth}{00 pt}
\setlength{\leftmargin}{0.0in}

%%%%%%%%%%%%%%%%%%%%%%%%%%%%%  Set Up The "Version" Package %%%%%%%%%%%%%%%%%%%%%%%%%%%%%%

% This environment allows one to include text in the LaTeX file that is not printed when it
% is compiled.

%%%%%%%%%%%%%%%%%%%%%%%%%%%%%%%% LOAD AMS LaTeX Packages %%%%%%%%%%%%%%%%%%%%%%%%%%%%%%%% 

% Using the "amsart" or "amsbook" document class instead of the standard LaTeX "article"
% document class provides all (or at least most) of the AMS math macros. However, it also
% puts a colon in the description list environment, which I don't like.

\usepackage{amsthm}
\usepackage{amsmath}
\usepackage{amssymb}
\usepackage{amsfonts}

% Added by EGP for script fonts such as script L for a linear operator: \mathscr{L}

\usepackage[mathscr]{euscript}

% This package provides an arrow \longrightarrow, that is longer than AMSMATH's
% \xrightarrow and \xrightarrow.

\usepackage{extarrows} % (extensible arrows)



%%%%%%%%%%%%%%%%%%%%%%%%%%%%%% SET UP THE "VERSION" PACKAGE %%%%%%%%%%%%%%%%%%%%%%%%%%%%%%

% This environment allows one to include text in the LaTeX file that is not printed when % it is compiled

% CAUTION: Do not use "space" as the name of a version environment, since it causes problems with LaTeX's memory

\usepackage{version}

\includeversion{SOLUTIONS}
\excludeversion{NO_SOLUTIONS}

\excludeversion{HIDE}



\usepackage{times}


\usepackage{wrapfig}

\usepackage{epsfig}
\usepackage{subfig}
\usepackage{ifpdf}
\usepackage{tikz}
\usetikzlibrary{shapes,calc}
\usepackage[utf8]{inputenc}
%\usepackage[upright]{fourier}
\usetikzlibrary{matrix,arrows,decorations.pathmorphing}

\usepackage[]{caption}
\setlength{\abovecaptionskip}{24pt}
\setlength{\belowcaptionskip}{24pt}

%%%%%%%%%%%%%%%%%%%%%%%%%%%%%%%% LOAD THE HYPERREF PACKAGE %%%%%%%%%%%%%%%%%%%%%%%%%%%%%%%

% The basic usage with the standard settings is straightforward. Just load the package in 
% the preamble, at the end of all the other packages but prior to other settings.

% From the `Introduction' to the hyperref manual: Make sure it comes last of your loaded 
% packages, to give it a fighting chance of not being over-written, since its job is to 
% redefine many LATEX commands. Hopefully you will find that all cross-references work 
% correctly as hypertext. For example, \section commands will produce a bookmark and a 
% link, whereas \section* commands will only show links when paired with a corresponding 
% \addcontentsline command.

% I think the default values for "bookmarks" is 'false' and for "bookmarksopen" is 'false'

\usepackage[bookmarksopen=false]{hyperref}

\hypersetup{colorlinks=true}

\hypersetup{linkcolor=blue}

\hypersetup{urlcolor=blue} 

\hypersetup{pdftitle={Harsha Lokavarapu's CV}}

\hypersetup{pdfcreator={\textcopyright Professor Elbridge Gerry Puckett 2018}}

\hypersetup{pdfauthor={Professor E.G. Puckett 2018}}

% This is a package to remember when writing NSF proposals.

% \usepackage[margin=1in]{geometry}

% And here is another one ...

%\usepackage{nopageno}

\usepackage{bbold}
\usepackage{wasysym}

\setlength{\parindent}{00pt}

\begin{document}

\begin{center}
  \textbf{Harsha Lokavarapu}      \\ [06pt]
  5221 Ferrera Ct                 \\
  Pleasanton, California 94588    \\
  lokavarapuh@gmail.com           \\  [3pt]
  GitHub address??
\end{center}

%\href{https://orcid.org/0000-0002-6589-7395}{ORCID ID}

\vskip 12pt

\begin{center}
	\textbf{\underline{Professional Preparation}}
\end{center}

\vskip -06pt

\begin{tabular}{llll}
University of California, Davis              &MS       &Computational Geodynamics (4.0 GPA)      & 2017-- \\
                                             &       &Thesis Adviser Louise H. Kellogg           &  \\
University of California, Davis              &BS      & Computer Science                         & 2015       \\
                                             &Minor   & Applied Mathematics                      & 2015
\end{tabular}

\vskip 18pt

\begin{center}
	\textbf{\underline{Appointments}}
\end{center}

\begin{tabular}{lll}
2014-2017     & \href{https://geodynamics.org/cig/}{Computational Infrastructure for Geodynamics} (CIG) & Junior Assistant Programmer \\
2012       & Certify Data Systems (Humana) & Internship as Code Developer
\end{tabular}

\vskip 18pt

\begin{center}
	\textbf{\underline{Programming Languages, Computing Skills, and Experience}}
\end{center}

\textbf{\underline{Languages}}
\begin{description}
  \item[] \textbf{\underline{Continuous Integration Tools}}
    \begin{description}
      \item[] Jenkins - touched every CIG code
      \item[] Travis
    \end{description}
  \item[] \textbf{\underline{Code Development}}
    \begin{description}
      \item Advanced Solver for Problems in Earth's ConvecTion (ASPECT)
      \item Calypso
      \item Generalized Reservoir Modeling (Ms. Thesis Project)
    \end{description}
  \item[] \\textbf{underline{Parallel Processing/High Performance Computing (HPC)}}
    \begin{description}
      \item National Science Foundation Texas Advanced Computing Center
      \begin{description}
        \item Stampede and Stampede 2.0 with Xeon Phi Processors
        \item Maverick - Nvidia K20 GPU cluster
      \end{description}
      \item Math and Physical Science (MPS) HPC Cluster
      \begin{description}
        \item Ymir \textcolor{red}{(Number of nodes, type of chip, RAM / node?) - lower priority but nice to have.}
        \item Peloton  \textcolor{red}{Same as  above ...}
      \end{description}
      \item SLURM
      \item Experience - ran strong and weak scaling tests for
      \begin{description}
        \item Calypso - published as poster at Fall AGU 2014
        \item ASPECT - As part of work associated with DSF paper (not included with published version)
      \end{description}
    \end{description}
  \item[] \underline{Outside Interests:}
    \begin{description}
      \item Virtual Reality - (A-frame)
      \item 3-D Design/Printing
      \item Kereas, Tensorflow 
    \end{description}
\end{description}

\vskip 12pt

\begin{center}
	\textbf{\underline{Professional Activities}}
\end{center}

\vskip -06pt

\addtolength{\tabcolsep}{15pt}   
\begin{tabular}{lll}
  2017--     & Member & ~\href{https://deepcarbon.net/}{Deep Carbon Observatory} \\
  2016     & Member & Annual Geophysical Union \\
  2015     & Member & Annual Geophysical Union \\
  2014     & Member & Annual Geophysical Union 
\end{tabular}
\addtolength{\tabcolsep}{1pt}  
 
\clearpage

\newpage

\begin{center}
  \textbf{\underline{Publications}}
\end{center}

\noindent
\textbf{\underline{Refereed Journal Publications}}

%\vskip 12pt

\noindent
\textbf{\underline{Submitted}}


\hangindent 20pt
L.~H.~Kellogg, D. L.~Turcotte, M.~Weisfeiler, $\textrm{H.~Lokavarapu}^@$, S.~Mukhopadhyay, (2018) 
``Implications of a Reservoir Model for the
Evolution of Deep Carbon'', 
\textit{Earth and Planetary Science Letters}, Ms. Ref. No.:  EPSL-D-17-01055

% \vskip 06pt

\noindent
\textbf{\underline{Accepted}}

% AUGUST 15, 2018 CHECKED!
\hangindent 20pt
R.~Gassmoeller,  $\textrm{H.~Lokavarapu}^@$, E.~Heien, E. G.~Puckett, and W.~Bangerth, (2018) 
``Flexible and scalable particle-in-cell methods with adaptive mesh refinement for geodynamic computations'', 
\textit{Geochemistry, Geophysics, Geosystems} manuscript 2018GC007508R 
\href{https://www.math.ucdavis.edu/~egp/PUBLICATIONS/JOURNAL_ARTICLES/ACCEPTED/RG-HL-EH-EGP-WB-2018.pdf}{View Accepted Manuscript}

\noindent
\textbf{\underline{Appeared}}

\vskip 06pt


% AUGUST 16, 2018 CHECKED!
\hangindent 20pt
E. G.~Puckett, D. L.~Turcotte, L.~H.~Kellogg,  $\mathrm{Y.~He}^{\dagger}$, $\mathrm{J.~M.~Robey}^{*}$, and 
$\mathrm{H.~Lokavarapu}^{@}$ (2018)
``New numerical approaches for modeling thermochemical convection in a compositionally stratified fluid'', 
Special issue of . \textit{Physics of the Earth and Planetary Interiors} associated with the 15th Studies of the Earth's Deep Interior (SEDI) Symposium (\textit{Phys. Earth. Planet. In.)} \textbf{276}:10–35, 10.1016/j.pepi.2017.10.004
\href{https://www.math.ucdavis.edu/~egp/PUBLICATIONS/JOURNAL_ARTICLES/APPEARED/2018/EGP-DLT-YH-HL-JMR-LHK-2018.pdf}{View Article}

\noindent
\textbf{\underline{Poster Presentations}}

% \vskip 12pt

\hangindent 20pt
L.~H.~Kellogg, $\mathrm{H.~Lokavarapu}^{@}$, D. L.~Turcotte, and S.~Mukhopadhyay (2017) 
``A reservoir model study of the flux of carbon from the atmosphere, to the continental crust, to the mantle'', 
\textit{Annual Geophysical Union Fall Meeting 2017}
\href{http://adsabs.harvard.edu/abs/2017AGUFMDI14A..06K}{View Abstract}

\hangindent 20pt
J.~Jiang, A.~P.~Kaloti, H.~R.~Levinson, N.~Nguyen, E.~G.~Puckett, and $\mathrm{H.~Lokavarapu}^{@}$ (2016) 
``Benchmark Results Of Active Tracer Particles In The Open Souce Code ASPECT For Modelling Convection In The Earth's Mantle'', 
\textit{Annual Geophysical Union Fall Meeting 2016}
\href{http://adsabs.harvard.edu/abs/2016AGUFM.T23C2946J}{View Abstract}

\hangindent 20pt
E.~G.~Puckett, D. L.~Turcotte, L.~H.~Kellogg, $\mathrm{H.~Lokavarapu}^{@}$,  $\mathrm{Y.~He}^{\dagger}$, and $\mathrm{J.~M.~Robey}^{*}$ (2016) 
``New Numerical Approaches To thermal Convection In A Compositionally Stratified Fluid'', 
\textit{Annual Geophysical Union Fall Meeting 2016}
\href{http://adsabs.harvard.edu/abs/2016AGUFMDI23A2589P}{View Abstract}

\hangindent 20pt
$\mathrm{H.~Lokavarapu}^{@}$, and H.~Matsui (2015) 
``Optimization of Parallel Legendre Transform using Graphics Processing Unit (GPU) for a Geodynamo Code'', 
\textit{Annual Geophysical Union Fall Meeting 2015}
\href{http://adsabs.harvard.edu/abs/2015AGUFMGP43B1253L}{View Abstract}

\hangindent 20pt
J.~A.~Russo, E.~H.~Studley, $\mathrm{H.~Lokavarapu}^{@}$, I.~Cherkashin, and E.~G.~Puckett (2014) 
``A New Monotonicity-Preserving Numerical Method for Approximating Solutions to the Rayleigh-Benard Equations'', 
\textit{Annual Geophysical Union Fall Meeting 2014}
\href{http://adsabs.harvard.edu/abs/2014AGUFMDI11A4258R}{View Abstract}

\hangindent 20pt
$\mathrm{H.~Lokavarapu}^{@}$, H.~Matsui, and E.~M.~Heien (2014) 
``Parallelization of the Legendre Transform for a Geodynamics Code'', 
\textit{Annual Geophysical Union Fall Meeting 2014}
\href{http://adsabs.harvard.edu/abs/2014AGUFMDI11A4255L}{View Abstract}

\vskip 18pt

$\phantom{0}^@$Undergraduate Student        \\
$\phantom{0}^*$Graduate Student             \\
$\phantom{0}^{\dagger}$Postdoctoral Scholar


\clearpage


\newpage

\begin{center}
  \textbf{\underline{Educational Details:}}
\end{center}

\noindent
\textbf{\underline{Math Courses}}
\begin{itemize} 
  \item 21B - Differential Calculus
  \item 21C - Integral Calculus
  \item 21D - Vector Analysis
  \item 22A - Linear Algebra
  \item 22B - Differential Equations
  \item 118A - Partial Differential Equations
  \item 118B
  \item 125A - Real Analysis (Foundations of Calculus)
  \item 125B
  \item 135A - Probability
  \item 150A - Modern Algebra
  \item 150B
  \item 167 - Advanced Linear Algebra: Matrix Methods in Data mining and Pattern Recognition
  \item 228A - Computational methods for Differential Equations
\end{itemize}

\noindent
\textbf{\underline{Computer Science Courses}}
\begin{itemize}
  \item 10 - Concepts of Computing
  \item 20 - Discrete Mathematics for Computer Science
  \item 30 - Introduction to Programming and Problem Solving
  \item 40 - Software and Object-Oriented Programming
  \item 50 - Machine Dependent Programming
  \item 60 - Data Structures and Programming
  \item 120 - Theory of Computation
  \item 122A - Algorithm Design
  \item 140A - Programming Languages
  \item 150 - Operating Systems
  \item 152A - Computer Networks
  \item 153 - Computer Security
  \item 154A - Computer Architecture
  \item 158 - Parallel Architectures
  \item 170 - Artificial Intelligence
  \item 188 - Ethics in an Age of Technology
\end{itemize}

\end{document}