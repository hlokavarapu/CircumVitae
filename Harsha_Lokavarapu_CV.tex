%        10        20        30        40        50        60       70        80        90
%23456789012345678901234567890123456789012345678901234567890123456789012345678901234567890123456

% \documentclass[11pt]{article} %\documentclass[final,10pt,reqno, oneside,dvipsnames]{article}

% WARNING: Use of 'twoside' will cause \marginpar to flip from right to left margin, depending
% on the page.

\documentclass[11pt]{ltxdoc}

%%%%%%%%%%%%%%%%%%%%%%%%%%%%%%%%%%%%%%%%%%%%%%%%%%%%%%%%%%%%%%%%%%%%%%%%%%%%%%%%%%%%%%%%%%%%%%%% Template provided by PROFESSOR ELBRIDGE GERRY PUCKETT
%%%%%%%%%%%%%%%%%%% Harsha Lokavarapu CURRICULUM VITAE %%%%%%%%%%%%%%%%%%%%%%%%
%%%%%%%%%%%%%%%%%%%%%%%%%%%%%%%%%%%%%%%%%%%%%%%%%%%%%%%%%%%%%%%%%%%%%%%%%%%%%%%%%%%%%%%%%%%%%%%%


% PROFESSOR ELBRIDGE GERRY PUCKETT
% DEPARTMENT OF MATHEMATICS
% UNIVERSITY OF CALIFORNIA, DAVIS
%
% Original LaTeX file created by
%
%   Wednesday, July 04, 2018 08:49:51 PDT
%
%
% REVISION HISTORY:
%
%   Revision 1.00
%
%     Revision 1.00: Wednesday, July   04, 2018 08:49:51 PDT

%
%
%%%%%%%%%%%%%%%%%%%%%%%%%%%%%%%%%% TURN OFF BibTeX? %%%%%%%%%%%%%%%%%%%%%%%%%%%%%%%%%%%%%%%%%%%%

% If there is no need for BibTeX in this document turn it off, since it creates unwanted 
% files.

\let\nobibtex = t

%%%%%%%%%%%%%%%%%%%%%%%%%%%%%%%%% SET UP PAGE PARAMETERS %%%%%%%%%%%%%%%%%%%%%%%%%%%%%%%%%%%%%%%

% EGP's preferred page style for notes, homework assignments, exams, etc.

% Margins, paragraph indents, space between paragraphs if any, etc. Good references 
% include page 85 of "The LaTeX Companion" by Frank Mittelbach and Michel Goossens and 
% page 260 of "Math Into LaTeX" by George Gratzer.

% Enlarge the width and height of the printed page

\setlength{\textwidth }{7.50 in}
\setlength{\textheight}{9.50 in} % \textheight     = 9.25 in

% Space between the end of the odd and even side margins and the beginning of the text.

\setlength{\oddsidemargin }{0.0 in}
\setlength{\evensidemargin}{0.0 in}

% The side margins are 1.0 inch plus \hoffset and the top margins is 1.0 inch plus 
% \voffset. According to "The LaTeX Companion" the default values are \hoffset = 00 pt 
% and \voffset = 00 pt.

\setlength{\hoffset}{-0.50 in}
\setlength{\voffset}{-0.50 in}

% If there is no header in this document, set each of the following values to zero.

% This is the amount of space between the BOTTOM of the HEAD and the TOP of the BODY. 
% According to "Math Into LaTeX"  by George Gratzer the default values for LaTeX's 
% article document is \headsep = 25 pt.

\setlength{\headsep}{12 pt}

% Package Fancyhdr Warning: \headheight = 12 pt is too small. Make it at least 14.0pt. 
% According to "Math Into LaTeX"  by George Gratzer the default value for LaTeX's article 
% document is \headheight = 12 pt.

\setlength{\headheight}{14 pt}

% This is the amount of space between the top of the page and the TOP of the header. According to "Math
% Into LaTeX"  by George Gratzer the default value for LaTeX's article document is \topmargin = 16 pt.
% \headheight = 12pt, and .

\setlength{\topmargin }{00 pt}

% This is the amount of space between the TOP OF THE BODY and THE BOTTOM OF THE FIRST 
% line of text on the page. According to "Math Into LaTeX" by George Gratzer the default
% value for LaTeX's article documents is \topskip = ?? pt.

\setlength{\topskip}{12 pt}

% This is the amount of space between the last line on the page and the footer. According 
% to "The LaTeX Companion" and "Math Into LaTeX" the default value is \footskip = 30 pt.

\setlength{\footskip}{18 pt}

% I like space between paragraphs, since it makes the document more readable. However, 
% this does not seem to change the spacing between paragraphs contained in an item of a 
% list.

\setlength{\parskip}{12 pt} % default: \parskip        = 12  pt

% Comment out the following line to use the default amount to indent the first line of 
% each paragraph.

\setlength{\parindent}{00 pt}

% Add an underline of text that breaks the line properly with \uline{text} ...

\usepackage[normalem]{ulem}

%%%%%%%%%%%%%%%%%%%%%%%%%%%%%%% LOAD THE "enumitem" PACKAGE %%%%%%%%%%%%%%%%%%%%%%%%%%%%%%

\usepackage{enumitem}

% \setenumerate[1]{labelindent=0pt,itemindent=12pt}

\setlength{\labelwidth}{00 pt}
\setlength{\leftmargin}{0.0in}

%%%%%%%%%%%%%%%%%%%%%%%%%%%%%  Set Up The "Version" Package %%%%%%%%%%%%%%%%%%%%%%%%%%%%%%

% This environment allows one to include text in the LaTeX file that is not printed when it
% is compiled.

%%%%%%%%%%%%%%%%%%%%%%%%%%%%%%%% LOAD AMS LaTeX Packages %%%%%%%%%%%%%%%%%%%%%%%%%%%%%%%% 

% Using the "amsart" or "amsbook" document class instead of the standard LaTeX "article"
% document class provides all (or at least most) of the AMS math macros. However, it also
% puts a colon in the description list environment, which I don't like.

\usepackage{amsthm}
\usepackage{amsmath}
\usepackage{amssymb}
\usepackage{amsfonts}

% Added by EGP for script fonts such as script L for a linear operator: \mathscr{L}

\usepackage[mathscr]{euscript}

% This package provides an arrow \longrightarrow, that is longer than AMSMATH's
% \xrightarrow and \xrightarrow.

\usepackage{extarrows} % (extensible arrows)

\newcommand{\tabitem}{~~\llap{\textbullet}~~}

%%%%%%%%%%%%%%%%%%%%%%%%%%%%%% SET UP THE "VERSION" PACKAGE %%%%%%%%%%%%%%%%%%%%%%%%%%%%%%

% This environment allows one to include text in the LaTeX file that is not printed when % it is compiled

% CAUTION: Do not use "space" as the name of a version environment, since it causes problems with LaTeX's memory

\usepackage{version}

\includeversion{SOLUTIONS}
\excludeversion{NO_SOLUTIONS}

\excludeversion{HIDE}



\usepackage{times}


\usepackage{wrapfig}

\usepackage{epsfig}
\usepackage{subfig}
\usepackage{ifpdf}
\usepackage{tikz}
\usetikzlibrary{shapes,calc}
\usepackage[utf8]{inputenc}
%\usepackage[upright]{fourier}
\usetikzlibrary{matrix,arrows,decorations.pathmorphing}

\usepackage[]{caption}
\setlength{\abovecaptionskip}{24pt}
\setlength{\belowcaptionskip}{24pt}

%%%%%%%%%%%%%%%%%%%%%%%%%%%%%%%% LOAD THE HYPERREF PACKAGE %%%%%%%%%%%%%%%%%%%%%%%%%%%%%%%

% The basic usage with the standard settings is straightforward. Just load the package in 
% the preamble, at the end of all the other packages but prior to other settings.

% From the `Introduction' to the hyperref manual: Make sure it comes last of your loaded 
% packages, to give it a fighting chance of not being over-written, since its job is to 
% redefine many LATEX commands. Hopefully you will find that all cross-references work 
% correctly as hypertext. For example, \section commands will produce a bookmark and a 
% link, whereas \section* commands will only show links when paired with a corresponding 
% \addcontentsline command.

% I think the default values for "bookmarks" is 'false' and for "bookmarksopen" is 'false'

\usepackage[bookmarksopen=false]{hyperref}

\hypersetup{colorlinks=true}

\hypersetup{linkcolor=blue}

\hypersetup{urlcolor=blue} 

\hypersetup{pdftitle={Harsha Lokavarapu's CV}}

\hypersetup{pdfcreator={\textcopyright Professor Elbridge Gerry Puckett 2018}}

\hypersetup{pdfauthor={Professor E.G. Puckett 2018}}

% This is a package to remember when writing NSF proposals.

% \usepackage[margin=1in]{geometry}

% And here is another one ...

%\usepackage{nopageno}

\usepackage{bbold}
\usepackage{wasysym}

\setlength{\parindent}{00pt}

\begin{document}

\begin{center}
  \textbf{Harsha Lokavarapu}                                            \\ [06pt]
  5221 Ferrera Ct                                                       \\
  Pleasanton, California 94588                                          \\
 \href{mailto:lokavarapuh@gmail.com}{lokavarapuh@gmail.com}             \\
  \href{https://github.com/hlokavarapu}{https://github.com/hlokavarapu} \\ [3pt]
\end{center}

%\href{https://orcid.org/0000-0002-6589-7395}{ORCID ID}

\vskip 12pt

\begin{center}
	\textbf{\underline{Education}}
\end{center}

\vskip -06pt

\begin{tabular}{llll}
University of California, Davis              &MS       &Computational Geodynamics (4.0 GPA)      & 2017-- \\
                                             &       &Thesis Adviser: Professor Louise H. Kellogg           &  \\
                                                                                          &       &Secondary Adviser: Professor Eldridge G. Puckett           &  \\
University of California, Davis              &BS      & Computer Science                         & 2015       \\
                                             &Minor   & Applied Mathematics                      & 2015
\end{tabular}

\vskip 18pt

\begin{center}
	\textbf{\underline{Appointments}}
\end{center}

\begin{tabular}{lll}
2014-2017     & \href{https://geodynamics.org/cig/}{Computational Infrastructure for Geodynamics} (CIG) & Junior Assistant Programmer \\
2012       & \href{https://www.humana.com}{Certify Data Systems (Humana)} & Internship as Code Developer
\end{tabular}

\vskip 18pt

\begin{center}
	\textbf{\underline{Programming Languages, Computing Skills, and Work Experience}}
\end{center}

  \begin{description}
  	
    \item[] \textbf{\underline{Open Source Code Development}}
    
      \begin{itemize}
      	
        \item Advanced Solver for Problems in Earth's ConvecTion (\href{https://github.com/geodynamics/aspect}{ASPECT}) - A parallel, extensible finite element code to simulate convection in both 2D and 3D models written in C++. \textcolor{red}{There is more. Parameter parsing?}
      
      \item State of-the-art model of the Earth's Geodynamo,
      \href{https://geodynamics.org/cig/software/calypso/}{Calypso} - FORTRAN and CUDA
      \textcolor{red}{Is the correct language for the GPU code?}
      \textcolor{green}{That's right, CUDA is a C++ library designed for Nvidia GPUs}
      
      \item Generalized Reservoir Modeling (\href{https://github.com/hlokavarapu/resecore.git}{MS Thesis Project}) - Python
            
    \end{itemize}
                                            
    \item[] \textbf{\underline{Parallel Processing and High Performance Computing}}
          
      \vskip 06pt
      
      \begin{description}
    
    	\item[] \textbf{Tools}
     
        \vskip 06pt   
        
        \begin{itemize} 
      
          \item SLURM HPC scheduler
                   
          \item Distributed memory parallelism - MPI for C++ and FORTRAN
      
          \item Shared memory parallelism - openMP
      
          \item CUDA - C++
      
          \item Profilers: gdb and cuda-gdb
      
      \end{itemize}
      
    \item[] \textbf{Machines}
    
      \vskip 06pt 
      
      \begin{itemize}
        
        \item National Science Foundation (NSF) Texas Advanced Computing Center
      
      \begin{itemize}
      	
        \item \href{https://portal.xsede.org/tacc-stampede}{Stampede} and \href{https://portal.xsede.org/tacc-stampede2}{Stampede 2}
        
        \item \href{https://portal.xsede.org/tacc-maverick}{Maverick} 
        
      \end{itemize}
  
      \item UCD Math and Physical Sciences (MPS) HPC Cluster
      
        \begin{itemize}
      	
          \item \href{https://wiki.cse.ucdavis.edu/support/systems/ymir}{Ymir} - 38 Dual socket, quad core (Intel E5620 2.4 GHz CPUs) with 24 GB RAM.
        
          \item \href{https://wiki.cse.ucdavis.edu/support/systems/peloton}{Peloton} - 55 nodes with 64GB ram, 16 cores / 32 threads (Intel Xeon E5-2630v3 Processors).
        
        \end{itemize}
  
      \end{itemize}  

    \item[] \textbf{Computations}
    
      \vskip 06pt
      

      \begin{itemize}
      	
        \item \href{https://geodynamics.org/cig/software/aspect/}{ASPECT}
              
          \begin{itemize}        
            
            \item Design and implement parallel particle generation algorithms
           
            \item Design and implement parallel particle interpolation algorithms including harmonic averaging and bilinear least squares
           
            \item Design 2D benchmarks to test the accuracy of particle algorithms in a finite element code
           
            \item Implement Schmeling subducting slab benchmark from Schmeling et al., Physics of the Earth and Planetary Interiors 171 (2008) 198--223
            
            \item Execute strong and weak scaling tests for original draft of publication [3] (see below), which was not included in the final publication              

           \end{itemize}

        \item       \href{https://geodynamics.org/cig/software/calypso/}{Calypso}         

          \begin{itemize}
          	
          	\item Optimization of Legendre Polynomial transform in spherical geometry using CUDA for Nvidia GPUs 
          	
          	\item Designed different implementations using CUDA Fast Fourier Transform 
          	  (\href{https://docs.nvidia.com/cuda/cufft/index.html}{cuFFT}) library, CUDA Basic Linear Algebra Subprograms (\href{https://docs.nvidia.com/cuda/cublas/index.html}{cuBLAS}), and \href{https://nvlabs.github.io/cub/}{CUB} library
          	  \textcolor{red}{Take out the links to (\href{https://docs.nvidia.com/cuda/cufft/index.html}{cuFFT}) library, CUDA Basic Linear Algebra Subprograms (\href{https://docs.nvidia.com/cuda/cublas/index.html}{cuBLAS}), and \href{https://nvlabs.github.io/cub/}{CUB}.
              You can leave the words in, but they want to see the code you wrote.
          	  WHERE IS THE CODE YOU WROTE????}
          	
          	\item Profile and test optimizations using strong and weak scaling tests
          	
          	\item Published results as poster [4], [6] (see below) at the 2014, 2015 Annual Fall AGU Meetings
          	
          \end{itemize}
           

      \end{itemize}

    
    \end{description}
    
    \item[] \textbf{\underline{Data Analysis and Visualization}}
    
      \begin{itemize}
      	    	
        \item R
       
        \item Python Libraries - matplotlib, numpy, scipy, and pandas
        
        \item Gnuplot
        
        \item Paraview 
        
        \item Visit
        
        
      \end{itemize}
%    \item[] \textbf{\underline{Continuous Integration Tools}}
  
%    \begin{itemize}
%      \item Jenkins - Java
%      \item Travis
%    \end{itemize}
    
    \item[] \textbf{\underline{Extracurricular Interests}}
    
      \begin{itemize}
        \item Virtual Reality - (\href{https://aframe.io/}{A-frame}) - JavaScript
        \item  3-D Design/Printing - (\href{https://www.tinkercad.com/#/}{Tinkercad})
      
        \item  Machine Learning - (\href{https://keras.io}{Keras}, \href{https://www.tensorflow.org/guide/premade_estimators}{Tensorflow}) - Python
        
        \item \href{https://github.com/jcjohnson/neural-style}{Neural style}
      
      \end{itemize}

  \end{description}


\vskip 12pt

\begin{center}
	\textbf{\underline{Professional Affiliations and Activities}}
\end{center}

\vskip -06pt

\addtolength{\tabcolsep}{15pt}   
\begin{tabular}{lll}
  2017--     & Member & \href{https://deepcarbon.net/}{Deep Carbon Observatory} \\[06pt]
  May 6-- May 17, 2017 & Participant & \href{https://geodynamics.org/cig/events/calendar/2017-aspect-hack/}{2017 ASPECT Hackathon} \\[06pt]
  2014 -- 2016     & Member & 
 \href{ https://sites.agu.org}{ American Geophysical Union (AGU)} \\[06pt]
  June 24 -- July 2, 2016     & Participant & \href{https://geodynamics.org/cig/events/calendar/2016-cig-all-hands-meeting/2016-aspect-hack/}{2016 ASPECT Hackathon} \\[06pt]
  June 17 -- June 24, 2016  & Staff & \href{https://geodynamics.org/cig/events/calendar/2016-cig-all-hands-meeting/}{CIG All Hands Meeting} \\[06pt]
  May 19-- May 30, 2015 & Participant & \href{https://geodynamics.org/cig/events/calendar/2015-aspect-hackathon/}{2015 ASPECT Hackathon} \\[06pt]
\end{tabular}
\addtolength{\tabcolsep}{1pt}  

\newpage

\begin{center}
  \textbf{\underline{Publications}}
\end{center}

\noindent
\textbf{\underline{Refereed Journal Publications}}

%\vskip 12pt

\noindent
\underline{Submitted}

\hangindent 20pt
[1]~L.~H.~Kellogg, D. L.~Turcotte, M.~Weisfeiler, $\textrm{H.~Lokavarapu}^@$, S.~Mukhopadhyay, (2018) 
``Implications of a Reservoir Model for the
Evolution of Deep Carbon'', 
\textit{Earth and Planetary Science Letters}, Ms. Ref. No.:  EPSL-D-17-01055

\noindent
\underline{Accepted}

\hangindent 20pt
[2]~R.~Gassmoeller,  $\textrm{H.~Lokavarapu}^@$, E.~Heien, E. G.~Puckett, and W.~Bangerth, (2018) 
``Flexible and scalable particle-in-cell methods with adaptive mesh refinement for geodynamic computations'', 
\textit{Geochemistry, Geophysics, Geosystems} manuscript 2018GC007508R 
\href{https://www.math.ucdavis.edu/~egp/PUBLICATIONS/JOURNAL_ARTICLES/ACCEPTED/RG-HL-EH-EGP-WB-2018.pdf}{View Accepted Manuscript}

\noindent
\underline{Appeared}

\vskip 06pt

\hangindent 20pt
[3]~E. G.~Puckett, D. L.~Turcotte, L.~H.~Kellogg,  $\mathrm{Y.~He}^{\dagger}$, $\mathrm{J.~M.~Robey}^{*}$, and 
$\mathrm{H.~Lokavarapu}^{@}$ (2018)
``New numerical approaches for modeling thermochemical convection in a compositionally stratified fluid'', 
Special issue of . \textit{Physics of the Earth and Planetary Interiors} associated with the 15th Studies of the Earth's Deep Interior (SEDI) Symposium (\textit{Phys. Earth. Planet. In.)} \textbf{276}:10–35, 10.1016/j.pepi.2017.10.004
\href{https://www.math.ucdavis.edu/~egp/PUBLICATIONS/JOURNAL_ARTICLES/APPEARED/2018/EGP-DLT-YH-HL-JMR-LHK-2018.pdf}{View Article}

\noindent
\textbf{\underline{Poster Presentations}}

% \vskip 12pt

\hangindent 20pt
[1]~L.~H.~Kellogg, $\mathrm{H.~Lokavarapu}^{@}$, D. L.~Turcotte, and S.~Mukhopadhyay (2017) 
``A reservoir model study of the flux of carbon from the atmosphere, to the continental crust, to the mantle'', 
\textit{Annual Geophysical Union Fall Meeting 2017}
\href{http://adsabs.harvard.edu/abs/2017AGUFMDI14A..06K}{View Abstract}

\hangindent 20pt
[2]~J.~Jiang, A.~P.~Kaloti, H.~R.~Levinson, N.~Nguyen, E.~G.~Puckett, and $\mathrm{H.~Lokavarapu}^{@}$ (2016) 
``Benchmark Results Of Active Tracer Particles In The Open Souce Code ASPECT For Modelling Convection In The Earth's Mantle'', 
\textit{Annual Geophysical Union Fall Meeting 2016}
\href{http://adsabs.harvard.edu/abs/2016AGUFM.T23C2946J}{View Abstract}

\hangindent 20pt
[3]~E.~G.~Puckett, D. L.~Turcotte, L.~H.~Kellogg, $\mathrm{H.~Lokavarapu}^{@}$,  $\mathrm{Y.~He}^{\dagger}$, and $\mathrm{J.~M.~Robey}^{*}$ (2016) 
``New Numerical Approaches To thermal Convection In A Compositionally Stratified Fluid'', 
\textit{Annual Geophysical Union Fall Meeting 2016}
\href{http://adsabs.harvard.edu/abs/2016AGUFMDI23A2589P}{View Abstract}

\hangindent 20pt
[4]~$\mathrm{H.~Lokavarapu}^{@}$, and H.~Matsui (2015) 
``Optimization of Parallel Legendre Transform using Graphics Processing Unit (GPU) for a Geodynamo Code'', 
\textit{Annual Geophysical Union Fall Meeting 2015}
\href{http://adsabs.harvard.edu/abs/2015AGUFMGP43B1253L}{View Abstract}

\hangindent 20pt
[5]~J.~A.~Russo, E.~H.~Studley, $\mathrm{H.~Lokavarapu}^{@}$, I.~Cherkashin, and E.~G.~Puckett (2014) 
``A New Monotonicity-Preserving Numerical Method for Approximating Solutions to the Rayleigh-Benard Equations'', 
\textit{Annual Geophysical Union Fall Meeting 2014}
\href{http://adsabs.harvard.edu/abs/2014AGUFMDI11A4258R}{View Abstract}

\hangindent 20pt
[6]~$\mathrm{H.~Lokavarapu}^{@}$, H.~Matsui, and E.~M.~Heien (2014) 
``Parallelization of the Legendre Transform for a Geodynamics Code'', 
\textit{Annual Geophysical Union Fall Meeting 2014}
\href{http://adsabs.harvard.edu/abs/2014AGUFMDI11A4255L}{View Abstract}

\vskip 18pt

$\phantom{0}^@$Undergraduate Student        \\
$\phantom{0}^*$Graduate Student             \\
$\phantom{0}^{\dagger}$Postdoctoral Scholar

\newpage

\begin{center}
  \textbf{\underline{CLASSES}}
\end{center}

\vskip -6pt

\begin{tabular}{lll}
  \textbf{\underline{Computer Science}} & \textbf{\underline{Mathematics}} \\[12pt]
  \tabitem 10 - Concepts of Computing & \tabitem 21A - Differential Calculus \\
  \tabitem 20 - Discrete Mathematics for Computer Science & 	\tabitem 21B - Integral Calculus \\
  \tabitem 30 - Introduction to Programming and Problem Solving & 	\tabitem 21C - Expansions, Series, etc. \\
  \tabitem 40 - Software and Object-Oriented Programming & 	\tabitem 21D - Vector Analysis \\
  \tabitem 50 - Machine Dependent Programming & 	\tabitem 22A - Linear Algebra \\
  \tabitem 60 - Data Structures and Programming & 	\tabitem 22B - Ordinary Differential Equations \\
  \tabitem 120 - Theory of Computation & 	\tabitem 118A - Partial Differential Equations   (first quarter) \\
  \tabitem 122A - Algorithm Design & 	\tabitem 118B - Partial Differential Equations   (second quarter) \\
  \tabitem 140A - Programming Languages & 	\tabitem 125A - Real Analysis (Foundations of Calculus) \\
  \tabitem 150 - Operating Systems & 	\tabitem 125B - Real Analysis (second quarter) \\
  \tabitem 152A - Computer Networks & 	\tabitem 135A - Probability \\
  \tabitem 153 - Computer Security & 	\tabitem 150A - Modern Algebra (first quarter) \\
  \tabitem 154A - Computer Architecture & 	\tabitem 150B - Modern Algebra (second quarter) \\
  \tabitem 158 - Parallel Architectures & 	\tabitem 167  - Advanced Linear Algebra: Machine Learning \\
  \tabitem 170 - Artificial Intelligence & 	\tabitem 228A - Computational methods for Partial Differential \\
  \tabitem 188 - Ethics in an Age of Technology & \qquad\qquad Equations
\end{tabular}

\begin{center}
	\textbf{\textcolor{red}{\underline{READ THESE NOTES}}}
\end{center}

  \begin{itemize}
  	
  	\item Harsha you need a lot more links to your own code; e.g., your code in R, Python, Jupyter notebooks, etc.
  	
  	\item If you don't have time to make a pile of links just put it all in your GitHub repo and let  
  	    them hunt around. 
  	    
	\item Having a link to someone else's GitHub repo won't be effective unless it is for your 
	    extracurricular activities \textbf{\textit{only}} such as \href{https://github.com/jcjohnson/neural-style}{Neural style}.
	
	\item \textcolor{red}{Can you point them to a place in your or another GitHub repo that has the   
		Cuda code?
        Anyone can say they wrote in Cuda, they want to see your code!.}
  
\end{itemize}



\end{document}