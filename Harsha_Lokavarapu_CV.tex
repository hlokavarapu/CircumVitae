\documentclass[11pt]{ltxdoc}

\let\nobibtex = t

\setlength{\textwidth }{7.50 in}
\setlength{\textheight}{9.50 in} % \textheight     = 9.25 in

\setlength{\oddsidemargin }{0.0 in}
\setlength{\evensidemargin}{0.0 in}


\setlength{\hoffset}{-0.50 in}
\setlength{\voffset}{-0.50 in}


\setlength{\headsep}{12 pt}


\setlength{\headheight}{14 pt}

\setlength{\topmargin }{00 pt}


\setlength{\topskip}{12 pt}


\setlength{\footskip}{18 pt}


\setlength{\parskip}{12 pt} 
\setlength{\parindent}{00 pt}

\usepackage[normalem]{ulem}


\usepackage{enumitem}

% \setenumerate[1]{labelindent=0pt,itemindent=12pt}

\setlength{\labelwidth}{00 pt}
\setlength{\leftmargin}{0.0in}

\usepackage{amsthm}
\usepackage{amsmath}
\usepackage{amssymb}
\usepackage{amsfonts}

\usepackage[mathscr]{euscript}


\usepackage{extarrows} % (extensible arrows)

\newcommand{\tabitem}{~~\llap{\textbullet}~~}

\usepackage{version}

\includeversion{SOLUTIONS}
\excludeversion{NO_SOLUTIONS}

\excludeversion{HIDE}



\usepackage{times}


\usepackage{wrapfig}

\usepackage{epsfig}
\usepackage{subfig}
\usepackage{ifpdf}
\usepackage{tikz}
\usetikzlibrary{shapes,calc}
\usepackage[utf8]{inputenc}
%\usepackage[upright]{fourier}
\usetikzlibrary{matrix,arrows,decorations.pathmorphing}

\usepackage[]{caption}
\setlength{\abovecaptionskip}{24pt}
\setlength{\belowcaptionskip}{24pt}


\usepackage[bookmarksopen=false]{hyperref}

\hypersetup{colorlinks=true}

\hypersetup{linkcolor=blue}

\hypersetup{urlcolor=blue} 

\hypersetup{pdftitle={Harsha Lokavarapu's CV}}

\hypersetup{pdfcreator={\textcopyright Professor Elbridge Gerry Puckett 2018}}

\hypersetup{pdfauthor={Professor E.G. Puckett 2018}}


%\usepackage{nopageno}

\usepackage{bbold}
\usepackage{wasysym}

\setlength{\parindent}{00pt}

\begin{document}

\begin{center}
  \textbf{Harsha Lokavarapu}                                            \\ [06pt]
  5221 Ferrera Court                                                    \\
  Pleasanton, CA 94588                                                  \\
  Mobile: (925) 998-5486                                                 \\
  Home:   (925) 416-0490                                                 \\
 \href{mailto:lokavarapuh@gmail.com}{lokavarapuh@gmail.com}             \\
  \href{https://github.com/hlokavarapu}{https://github.com/hlokavarapu} \\ [3pt]
\end{center}

%\href{https://orcid.org/0000-0002-6589-7395}{ORCID ID}

\vskip 12pt

\begin{center}
	\textbf{\underline{Education}}
\end{center}

\vskip -06pt

\begin{center}
\begin{tabular}{llll}
University of California, Davis              &MS       &Computational Geodynamics      & 2017-- \\
                                             &       &Thesis Adviser: \href{http://geology.ucdavis.edu/people/faculty/kellogg.php}{Professor Louise H. Kellogg}        &  \\
University of California, Davis              &BS      & Computer Science                         & 2015       \\
                                             &Minor   & Applied Mathematics                      & 2015
\end{tabular}
\end{center}
\vskip 18pt

\begin{center}
	\textbf{\underline{Appointments}}
\end{center}

\begin{center}
\begin{tabular}{lll}
2014-2017     & \href{https://geodynamics.org/cig/}{Computational Infrastructure for Geodynamics} (CIG) & Junior Assistant Programmer \\
2012       & \href{https://www.humana.com}{Certify Data Systems (Humana)} & Internship as Code Developer
\end{tabular}
\end{center}

\vskip 18pt

\begin{center}
	\textbf{\underline{Programming Languages, Computing Skills, and Work Experience}}
\end{center}

% * Note that all "Git" links are personal code samples 

  \begin{description}
 
    \item[] \textbf{\underline{Data Analysis and Visualization}}
    
      \begin{itemize}
      	\item I have worked with computing numerical models on several super computers housed at TACC. Several of these computational experiments generated datasets on the order of terabytes. Using numerical and visualization libraries in python, I have created several easy to share Jupyter notebooks to obtain numerical measures such as computational accuracy, stability, Kolmogorov Entropy, and others. Data formats that I am familiar with include HDF5, VTK, NETCDF and ASCII - (Git: \href{https://github.com/hlokavarapu/Prospectus/blob/master/SlideDeck/MeetingPresentation.ipynb}{Deep Carbon}, \href{https://github.com/EGP-CIG-REU/SECOND_PARTICLE_PAPER/blob/master/00JUPYTER_NOTEBOOKS/egp_time_dependent_annulus_v-02/egp_time_dependent_annulus_v-02-Exact.ipynb}{Convergence Analysis}, and \href{https://github.com/hlokavarapu/Geology_Journal/tree/master/00JUPYTER}{more})
        
        \item In cases where the underlying dataset to analyze is less than several gigabytes, I have come to rely on using programming language R and its rich statistical libraries. The ease of use to write small scripts to quickly calculate statistical measures has moved me to prefer R over other remaining languages - (Git: \href{https://github.com/hlokavarapu/computational_tools/tree/master/R_scripts}{Code})
        
        \item When dealing with extremely small datasets with the goal of quickly generating plots for discussion, I have come to rely on Gnuplot and its command line interface. Lately, I have been prefering python's matplotlib.pyplot module however - (Git: \href{https://github.com/hlokavarapu/computational_tools/tree/master/Gnuplot_scripts}{Code})
        
        \item Working with supercomputers with a predefined limit on core hour usage, I have developed experience making the most of the limited resources using various parallelization tricks. Many of these tricks require extensive experience with bash scripting in order to run numerous jobs, monitor the state of jobs, and data collection towards the end - (Git: \href{https://github.com/hlokavarapu/computational_tools/tree/master/slurm_scripts}{Code})
      	                
      \end{itemize}
    
    \item[] \textbf{\underline{Deep Learning}}
      \begin{itemize}
        \item Participant of Kaggle competitions
        \item Google cloud computing to train neural networks in a multi-GPU environment.
        \item[] \textbf{\underline{Coursera Certifications:}}
        \begin{itemize}
          \item Improving Deep Neural Networks: Hyperparameter tuning, Regularization and Optimization
          \item Neural Networks and Deep Learning
        \end{itemize}
      \end{itemize}
      
    \item[] \textbf{\underline{GPU Experience}}
    \begin{itemize}
    		\item In order to resolve MHD dynamo of planets and stars, it is of utmost importance to resolve small scales. Due to the underlying mathematical principles, physical quantites of interest were transformed into spherical harmonic coefficients, whose computational complexity is O(N$^3$). Using CUDA in a multi-GPU environment, I pursed the optimization of spherical harmonic transform (SHT). Optimizations involved taking advantage of state of the art GPU libraries: CUDA Fast Fourier Transform (cuFFT) library, CUDA Basic Linear Algebra Subprograms (cuBLAS), and (CUB) library. This alogirthm was optimized to a time complexity of O(N$^2$logN) - \href{https://github.com/hlokavarapu/calypso/tree/concurrency/src/Cuda_libraries}{Dev branch})
          	\item Profile and test optimizations using strong and weak scaling tests        	
          	\item Published results as poster [4], [6] (see below) at the 2014, 2015 Annual Fall AGU Meetings
    \end{itemize}
    
    \item[] \textbf{\underline{Supercomputing Experience}}
    
      \vskip 06pt 
      
      \begin{itemize}
        
        \item National Science Foundation (NSF) Texas Advanced Computing Center
      
      \begin{itemize}
      	
        \item \href{https://portal.xsede.org/tacc-stampede}{Stampede} and \href{https://portal.xsede.org/tacc-stampede2}{Stampede 2} - Ranked as 17th fastest supercomputer in top 500 list. Primarily utilized for runnining massive compuations using ASPECT to track interfaces in a multilayered buoyancy model over thousands of cores. Results and analysis were published in [1].
        
        \item \href{https://portal.xsede.org/tacc-maverick}{Maverick} - A multi-gpu Nvidia K20 cluster. Both development and analysis of optimzation of SHT was primarily conducted on this cluster. 
        
      \end{itemize}
     \end{itemize}
    
    \item[] \textbf{\underline{Open Source Code Development}}
    
      \begin{itemize}
      	
        \item Advanced Solver for Problems in Earth's ConvecTion (\href{https://github.com/geodynamics/aspect}{ASPECT}) written in C++. I am one of the active contributors of this project.
        
            \begin{itemize}        
            
            \item Design and implementation of parallel particle generation algorithms. (Git: \href{https://github.com/geodynamics/aspect/pull/1266}{PR})
           
            \item Design and implementation of parallel particle interpolation algorithms including harmonic averaging and bilinear least squares. (Git PR: \href{https://github.com/geodynamics/aspect/pull/1949}{Harmonic Average}, \href{https://github.com/geodynamics/aspect/pull/1554}{Bilinear least squares})
           
            \item Design 2-D analytical solution to Stokes equations in order to benchmark the accuracy of particle algorithms in ASPECT. (Git PR: \href{https://github.com/EGP-CIG-REU/aspect/tree/EGP_HVL_benchmark/benchmark/egp_hvl}{1}, \href{https://github.com/EGP-CIG-REU/aspect/tree/simple_annulus_benchmark/benchmarks/simple_annulus}{2})
                       
            \item Execute strong and weak scaling tests for original draft of publication [3] (see below), which was not included in the final publication              
            
            \item ASPECT contributions Git \href{https://github.com/geodynamics/aspect/commits?author=hlokavarapu}{timeline}

           \end{itemize}
      
      \item State of-the-art model of the Earth's Geodynamo,
      \href{https://geodynamics.org/cig/software/calypso/}{Calypso} written in FORTRAN 90. I am the 3/4 most active contributor.
      
         \begin{itemize}
          	\item Calypso contributions Git \href{https://github.com/geodynamics/calypso/commits/Legendre_transform_w_symmetry}{timeline}
          \end{itemize}
      
      \item Generalized reservoir modeling library (MS Thesis Project: \href{https://github.com/hlokavarapu/resecore.git}{Resecore}) written in Python
            
    \end{itemize}
                                      
    
  

    \item[] \textbf{\underline{Tools}}
    
      \begin{itemize}
        \item Paraview (Git: \href{https://github.com/hlokavarapu/computational_tools/blob/master/Paraview_scripts/Find_timestep_given_nondim_time_2_comp.py}{Code})

        \item Visit
        
          \item CMake, CTest - Build tool and Unit testing - (Git PR: \href{https://github.com/hlokavarapu/computational_tools/tree/master/ASPECT_build_scripts}{1}, \href{https://github.com/geodynamics/calypso/pull/4}{2})
      
          \item CUDA - (Git: \href{https://github.com/geodynamics/calypso/pull/3}{PR})(SHT = Spherical Harmonic Transform)
                   
          \item Distributed memory parallelism - MPI for C++ and FORTRAN
      
          \item Shared memory parallelism - openMP
          
          \item SLURM HPC scheduler
          
          \item Profilers: gdb and cuda-gdb
      \end{itemize}
      
  \item[] \textbf{\underline{Outside Interests}}
    
      \begin{itemize}
        \item Virtual Reality - (Git: \href{https://github.com/hlokavarapu/VR-Experiments}{VR-Experiments} using A-Frame library) - JavaScript
        \item  3-D Design/Printing - Tinkercad
      \end{itemize}
  \end{description}
  

%\newpage

\vskip 12pt

\begin{center}
  \textbf{\underline{Publications}}
\end{center}

\noindent
\textbf{\underline{Refereed Journal Publications}}

%\vskip 12pt

\noindent
\underline{Submitted}

\hangindent 20pt
[1]~L.~H.~Kellogg, D. L.~Turcotte, M.~Weisfeiler, $\textrm{H.~Lokavarapu}^*$, S.~Mukhopadhyay, (2018) 
``Implications of a Reservoir Model for the
Evolution of Deep Carbon'', 
\textit{Earth and Planetary Science Letters}, Ms. Ref. No.:  EPSL-D-17-01055

\noindent
\underline{Accepted}

\hangindent 20pt
[2]~R.~Gassmoeller,  $\textrm{H.~Lokavarapu}^*$, E.~Heien, E. G.~Puckett, and W.~Bangerth, (2018) 
``Flexible and scalable particle-in-cell methods with adaptive mesh refinement for geodynamic computations'', 
\textit{Geochemistry, Geophysics, Geosystems} manuscript 2018GC007508R 
\href{https://www.math.ucdavis.edu/~egp/PUBLICATIONS/JOURNAL_ARTICLES/ACCEPTED/RG-HL-EH-EGP-WB-2018.pdf}{View Accepted Manuscript}

\noindent
\underline{Appeared}

\vskip 06pt

\hangindent 20pt
[3]~E. G.~Puckett, D. L.~Turcotte, L.~H.~Kellogg,  $\mathrm{Y.~He}^{\dagger}$, $\mathrm{J.~M.~Robey}^{*}$, and 
$\mathrm{H.~Lokavarapu}^{@}$ (2018)
``New numerical approaches for modeling thermochemical convection in a compositionally stratified fluid'', 
Special issue of . \textit{Physics of the Earth and Planetary Interiors} associated with the 15th Studies of the Earth's Deep Interior (SEDI) Symposium (\textit{Phys. Earth. Planet. In.)} \textbf{276}:10–35, 10.1016/j.pepi.2017.10.004
\href{https://www.math.ucdavis.edu/~egp/PUBLICATIONS/JOURNAL_ARTICLES/APPEARED/2018/EGP-DLT-YH-HL-JMR-LHK-2018.pdf}{View Article}

\noindent
\textbf{\underline{Poster Presentations}}

% \vskip 12pt

\hangindent 20pt
[1]~L.~H.~Kellogg, $\mathrm{H.~Lokavarapu}^{*}$, D. L.~Turcotte, and S.~Mukhopadhyay (2017) 
``A reservoir model study of the flux of carbon from the atmosphere, to the continental crust, to the mantle'', 
\textit{Annual Geophysical Union Fall Meeting 2017}
\href{http://adsabs.harvard.edu/abs/2017AGUFMDI14A..06K}{View Abstract}

\hangindent 20pt
[2]~J.~Jiang, A.~P.~Kaloti, H.~R.~Levinson, N.~Nguyen, E.~G.~Puckett, and $\mathrm{H.~Lokavarapu}^{@}$ (2016) 
``Benchmark Results Of Active Tracer Particles In The Open Souce Code ASPECT For Modelling Convection In The Earth's Mantle'', 
\textit{Annual Geophysical Union Fall Meeting 2016}
\href{http://adsabs.harvard.edu/abs/2016AGUFM.T23C2946J}{View Abstract}

\hangindent 20pt
[3]~E.~G.~Puckett, D. L.~Turcotte, L.~H.~Kellogg, $\mathrm{H.~Lokavarapu}^{@}$,  $\mathrm{Y.~He}^{\dagger}$, and $\mathrm{J.~M.~Robey}^{*}$ (2016) 
``New Numerical Approaches To thermal Convection In A Compositionally Stratified Fluid'', 
\textit{Annual Geophysical Union Fall Meeting 2016}
\href{http://adsabs.harvard.edu/abs/2016AGUFMDI23A2589P}{View Abstract}

\hangindent 20pt
[4]~$\mathrm{H.~Lokavarapu}^{@}$, and H.~Matsui (2015) 
``Optimization of Parallel Legendre Transform using Graphics Processing Unit (GPU) for a Geodynamo Code'', 
\textit{Annual Geophysical Union Fall Meeting 2015}
\href{http://adsabs.harvard.edu/abs/2015AGUFMGP43B1253L}{View Abstract}

\hangindent 20pt
[5]~J.~A.~Russo, E.~H.~Studley, $\mathrm{H.~Lokavarapu}^{@}$, I.~Cherkashin, and E.~G.~Puckett (2014) 
``A New Monotonicity-Preserving Numerical Method for Approximating Solutions to the Rayleigh-Benard Equations'', 
\textit{Annual Geophysical Union Fall Meeting 2014}
\href{http://adsabs.harvard.edu/abs/2014AGUFMDI11A4258R}{View Abstract}

\hangindent 20pt
[6]~$\mathrm{H.~Lokavarapu}^{@}$, H.~Matsui, and E.~M.~Heien (2014) 
``Parallelization of the Legendre Transform for a Geodynamics Code'', 
\textit{Annual Geophysical Union Fall Meeting 2014}
\href{http://adsabs.harvard.edu/abs/2014AGUFMDI11A4255L}{View Abstract}

\vskip 18pt

$\phantom{0}^@$Undergraduate Student        \\
$\phantom{0}^*$Graduate Student             \\
$\phantom{0}^{\dagger}$Postdoctoral Scholar



\end{document}
