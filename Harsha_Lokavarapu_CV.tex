\documentclass[11pt]{ltxdoc}

\let\nobibtex = t

\setlength{\textwidth }{7.50 in}
\setlength{\textheight}{9.50 in} % \textheight     = 9.25 in

\setlength{\oddsidemargin }{0.0 in}
\setlength{\evensidemargin}{0.0 in}


\setlength{\hoffset}{-0.50 in}
\setlength{\voffset}{-0.50 in}


\setlength{\headsep}{12 pt}


\setlength{\headheight}{14 pt}

\setlength{\topmargin }{00 pt}


\setlength{\topskip}{12 pt}


\setlength{\footskip}{18 pt}


\setlength{\parskip}{12 pt} 
\setlength{\parindent}{00 pt}

\usepackage[normalem]{ulem}


\usepackage{enumitem}

% \setenumerate[1]{labelindent=0pt,itemindent=12pt}

\setlength{\labelwidth}{00 pt}
\setlength{\leftmargin}{0.0in}

\usepackage{amsthm}
\usepackage{amsmath}
\usepackage{amssymb}
\usepackage{amsfonts}

\usepackage[mathscr]{euscript}


\usepackage{extarrows} % (extensible arrows)

\newcommand{\tabitem}{~~\llap{\textbullet}~~}

\usepackage{version}

\includeversion{SOLUTIONS}
\excludeversion{NO_SOLUTIONS}

\excludeversion{HIDE}



\usepackage{times}


\usepackage{wrapfig}

\usepackage{epsfig}
\usepackage{subfig}
\usepackage{ifpdf}
\usepackage{tikz}
\usetikzlibrary{shapes,calc}
\usepackage[utf8]{inputenc}
%\usepackage[upright]{fourier}
\usetikzlibrary{matrix,arrows,decorations.pathmorphing}

\usepackage[]{caption}
\setlength{\abovecaptionskip}{24pt}
\setlength{\belowcaptionskip}{24pt}

\usepackage[bookmarksopen=false]{hyperref}

\hypersetup{colorlinks=true}

\hypersetup{linkcolor=blue}

\hypersetup{urlcolor=blue} 

\hypersetup{pdftitle={Harsha Lokavarapu's CV}}

\hypersetup{pdfcreator={\textcopyright Professor Elbridge Gerry Puckett 2018}}

\hypersetup{pdfauthor={Professor E.G. Puckett 2018}}


%\usepackage{nopageno}

\usepackage{bbold}
\usepackage{wasysym}

\usepackage{fancyhdr}
\pagestyle{fancy}
\rhead{(925)-998-5486 \\ lokavarapuh@gmail.com}
\lhead{\href{https://github.com/hlokavarapu}{https://github.com/hlokavarapu}}
\setlength{\parindent}{00pt}

\begin{document}

\begin{center}
  \textbf{Harsha Lokavarapu}
\end{center}
\vskip -10pt

\textbf{\underline{Education}}

\vskip -06pt

\begin{tabular}{llll}
University of California, Davis              &MS       &Computational Geodynamics      & 2018 \\
                                             &       &Thesis Advisor: \href{http://geology.ucdavis.edu/people/faculty/kellogg.php}{Professor Louise H. Kellogg}        &  \\
University of California, Davis              &BS      & Computer Science                         & 2015       \\
                                             &Minor   & Applied Mathematics                      & 2015
\end{tabular}

% * Note that all "Git" links are personal code samples 

\vskip -10pt

\textbf{\underline{Employment}}

		{\textcolor{teal}{Personal Projects} \hfill \textcolor{teal}{Kaggle - Machine Learning Competitions} \hfill \textcolor{teal}{2019-Present}}
		\begin{itemize}
        	  \item House Prices: Advanced Regression Techniques
        	    \begin{itemize}
        	      \item Use of python library Keras to develop a regression model
        	      \item Employed feature selection methods, feature crosses, normalization of data input, removal of outliers, and transformation of input data to correct for heavy left skew distributions
        	      \item In combination with tensorboard, executed a hyperparameter search to discover optimal number of fully connected layers, number of neurons per layer, activation functions, and learning rates
        	      \item Currently placed in the 34th percentile within the competition
        	    \end{itemize}
        	  \item Human Protein Atlas Image Classification
        	    \begin{itemize}
        	      \item This competitions' challenges included multilabel classification, 4 channel input images, and imbalanced dataset. Less than 1\% of the training image data consisted of 5 classes. Of the remaining classes, 3-4 classes were represented by 70\% of the data set
        	      \item My submitted model took advantage of input images resized to 128 by 128, focal loss function, data augmentation, external data, and oversampling to counteract the imbalanced dataset
        	      \item Wrote custom metrics and analyzed results on tensorboard in order to better measure the accuracy of various experiments
        	      \item Experiments included transfer learning to VGG network, different network architectures such as Conv2D sequential architecture, and Network in Network architectures and more
        	      \item Trained neural networks using Google cloud computing in a multi-GPU environment
        	    \end{itemize}
        	\end{itemize}
        	
		{\textcolor{teal}{Software Engineer} \hfill \textcolor{teal}{Uber} \hfill \textcolor{teal}{2019-2020}}
		\begin{itemize}
			\item Developed vInfra, a system to generate virtual infrastructure using an emulator
			\begin{itemize}
				\item used to generate Uber's on-prem backbone
				\item used by network ops to investigate network failures
			\end {itemize}
			
			\item Developed IRS, a golang command line tool to track inventory and reserve assets
			\begin{itemize}
				\item designed Service Now database schema to store reservations
				\item authored scripted endpoint in Service Now to handle REST API requests
				\item developed and implemented Thrift and gRPC API for reserving assets
			\end {itemize}
			
			\item Developed Metere, a repository for test case definitions and test case results
		 	\begin{itemize}
		 	  \item designed database schema using SQL
		 	  \item developed and implemented thrift API to create, read, update and delete test case definitions and results
		 	  \item developed front end application to interact with test case definitions and results repository using React and Base Web
		 	\end{itemize}
	 	\end{itemize}

	{\textcolor{teal}{Junior Assistant Programmer} \hfill \textcolor{teal}{Computational Infrastructure for Geodynamics} \hfill \textcolor{teal}{2014-2017}}

      \begin{itemize}
      	
        \item Contributed to open-source numerical library \href{https://github.com/geodynamics/aspect}{ASPECT} written in C++
        
            \begin{itemize}        
            	\item implemented parallel particle generation algorithms
            	\item implemented parallel particle interpolation algorithms including harmonic averaging and bilinear least squares
            	\item designed 2-D analytical solution to Stokes equations and benchmarked the accuracy of particle algorithms
			\end{itemize}
      
		\item Contributed to open-source numerical library,
		\href{https://geodynamics.org/cig/software/calypso/}{Calypso} written in Fortran 90.
  
         	\begin{itemize}
          		\item implemented and optimized spherical harmonic transform using cutting-edge GPU hardware with CUDA in C++
          		\item executed strong and weak scaling tests to measure performance on supercomputer Maverick
          	\end{itemize}
        \item Data analysis and automation with python
        	\begin{itemize}
        		\item created data pipelines for large data transfers from cluster to cluster
        		\item authored scripts to compute entropy as a funtion of time using particle positions 
        		\item implemented carbon reservoir model with interactive widgets to help scientists analyze the influence of different initial parameter configurations on the evolution of carbon cycle 
        	\end{itemize}            
	\end{itemize}
    
	\textcolor{teal}{Software Developer, Intern} \hfill \textcolor{teal}{Humana} \hfill \textcolor{teal}{2012}
		
		\begin{itemize}
			\item Wrote puppet manifests to install Humana application Healthdock's software stack for clients
			\item Written Puppet manifests include Apache Web Server, Tomcat, Oracle, Avahi, Samba, Java, Apelon, and Healthdock
		\end{itemize}
		
	\textbf{\underline{Coursera Certifications:}}
		\begin{itemize}
		  \item Convolutional Neural Networks
		  \item Improving Deep Neural Networks: Hyperparameter tuning, Regularization and Optimization
		  \item Neural Networks and Deep Learning
		\end{itemize}
	
\textbf{\underline{Skills:}} C++, CUDA, Distributed Services, Git, Go, gRPC, Python, Unix \\
(\emph{familiar with}): keras, ipywidgets, numpy


\end{document}
